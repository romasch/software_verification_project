\documentclass[a4paper,10pt]{article}
\usepackage[utf8]{inputenc}

% Handmade package to define patterns.
% \usepackage{pattern}
% For the \todo{} command.
\usepackage{todo}

% Nice fonts
\usepackage{palatino}
% Needed for Listings package with Eiffel.
\usepackage{xcolor}
% Source code listings.
\usepackage{listings}
% Appendix with extra title.
\usepackage [page] {appendix}
% To include PNG files.
\usepackage{graphicx}
% Nice looking captions.
\usepackage[font={footnotesize,sl}, labelfont=bf] {caption}
% Include PDF pages.
\usepackage{pdfpages}

% Clickable links. Has to be the last package:
\usepackage [hidelinks] {hyperref}


\lstset{language=OOSC2Eiffel,basicstyle=\ttfamily\small}
\definecolor{codebg}{rgb}{0.95,0.95,0.95}
\setlength{\headheight}{28pt}
\lstset{escapechar=\$}

\newcommand{\dir}{\emph}
\newcommand{\todoref}{\todo{ref}}

% Title Page
\title{Software Verification 2014 \\ Project Report}
\author{Roman Schmocker \\ 09-911-215}


\begin{document}

\maketitle


\section{Introduction}
This report covers the implementation and verification of the two sort algorithms BucketSort \todoref and QuickSort \todoref.
The tools used to verify the code are AutoProof \todoref and Boogie\todoref.

The structure of the report follows the chronological order in which the tasks were completed.
The first section deals with the Boogie verification of the Quicksort algorithm.
Following that is a section about the AutoProof solution.
The last section compares the two tools and draws conclusions.

\section {Boogie}

In the Boogie part I only verified Quicksort due to time constraints.
A particular challenge was to show that the elements of the final array are the same as in the input array.
An extensive literature study (i.e. typing ``boogie sort algorithm'' in Google) showed that the solution is to 
keep track of the permutation explicitly.

Therefore I chose to use three global variables:
\begin{itemize}
 \item The input array.
 \item The sorted output array.
 \item A permutation array to map the sorted array to the input array.
\end{itemize}

Using this system I was able to set up a global invariant that the permutation is always a valid mapping.
This invariant can be found in basically every precondition, postcondition and loop invariant, and it's the reason why the permutation proof became very simple in the end.

Proving that the output is actually sorted was another challenge.
Even though I could prove that each element in the sorted array maps to the input array,
Boogie was not able to infer that the corresponding minima and maxima in a subarray stay the same.

The solution was therefore to keep track of these values explicitly.
I had to extend the interface of the quicksort routine to get the boundary values and add preconditions, 
loop invariants and postconditions to make sure everything is maintained.


\section{Eiffel}

\subsection{Implementation}

The implementation of the features in \lstinline!SV_LIST! was pretty straightforward.
The problem however was, as I later realized, that the implementation wasn't always suited to verification.
Therefore the relevant sort features like Quicksort and Bucketsort had to be adapted several types, 
and the final implementation carries a lot of overhead to keep track of state needed for verification.

\subsection{Specification}

Due to my experience with Eiffel programs, specifying pre- and postconditions was very easy.
The only challenge were the sorting algorithms, where the specification should not just include that the array is sorted,
but also that the final array contains the same elements as the input array.
Otherwise one could just return a zero-filled array as a valid result.

The verification phase however brought some unplanned changes again.
I had to add some specifications, such as the number of elements staying the same during a sort, 
as AutoProof wasn't able to infer that from the fact that the output is a permutation of the input.

\subsection{Verification}

\textbf{General featuers}
The class \lstinline!SV_LIST! was shipped with many small features such as \lstinline!extend! or \lstinline!count!.
These were very easy to prove, often not even requiring any further specification besides pre- and postconditions.

\textbf{Quicksort}
The verification of Quicksort in AutoProof was very tough.
In fact during my first attempt I stopped with the AutoProof solution and started working on Boogie.
Thanks to the Boogie solution I then had a slightly better idea what was going on in the background.

The main problem was that the verifier had its own peculiar ideas about what is a permutation and what is not.
This is something which is still not resolved in the final solution.
While I was able to verify that the input is \emph{equal} to a local array, and the local array is a permutation of the sorted array, 
AutoProof cannot draw the conclusion that therefore the sorted array is a permutation of the input array.
This was the point where I gave up and added an assume clause.

I also encountered the same problems in the Eiffel as in Boogie, like the problem with minimum/maximum staying the same in a permtuted array.
As I already identified the problems earlier in Boogie I just implemented the same workarounds in Eiffel.

\textbf{Bucketsort}

\textbf{Array concatenation}

\section {Conclusion}

During this project I've used both AutoProof and Boogie to verify different sorting algorithm.
In my opinion, Boogie is a bit easier to use, because \todo{a reason}.

AutoProof on the other hand has a higher level of abstraction.
However I think that precisely this abstraction makes it very hard to use AutoProof, 
because when something goes wrong you often have no idea and no hint about what's the problem.
Another problem which mostly happened in AutoProof is that I often got a timeout error, which can be very frustrating.

An advantage of AutoProof on the other hand is that it nicely integrates into the Eiffel programming language.
This allows to at least partially prove some classes and features in a convenient and mostly automatic way, 
and use traditional testing for features that could not be verified.

In both tools there's very little help provided when debugging a failed assertion.
Getting it to verify therefore often involves hours of trial-and-error programming.
I think for a static verifier to succeed in practice it needs at least to provide a counter example 
or to say why it couldn't verify a statement, but this might be impossible to implement.

In my opinion static verification of programs is not usable in practice.
It can be very hard to prove even simple programs and checks where programmers can easily reason that a program is correct.
Furthermore, there's the problem that the specification itself may be wrong, and the verifier happily verifies a buggy program.
I think time and money is better invested into rigorous testing and traditional mechanisms to ensure high quality, such as code reviews.

\todos

\begin{flushleft}
{{{
\bibliographystyle {plain}
\bibliography {./references}
}}}
\end{flushleft}

\end{document}          
 
