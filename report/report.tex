\documentclass[a4paper,10pt]{article}
\usepackage[utf8]{inputenc}


% For the \todo{} command.
\usepackage{todo}

% Nice fonts
\usepackage{palatino}
% Needed for Listings package with Eiffel.
\usepackage{xcolor}
% Source code listings.
\usepackage{listings}
% Appendix with extra title.
% \usepackage [page] {appendix}
% To include PNG files.
% \usepackage{graphicx}
% Nice looking captions.
% \usepackage[font={footnotesize,sl}, labelfont=bf] {caption}
% Include PDF pages.
% \usepackage{pdfpages}

% Clickable links. Has to be the last package:
\usepackage [hidelinks] {hyperref}


\lstset{language=OOSC2Eiffel,basicstyle=\ttfamily\small}
\definecolor{codebg}{rgb}{0.95,0.95,0.95}
\setlength{\headheight}{28pt}
\lstset{escapechar=\$}

\newcommand{\dir}{\emph}
\newcommand{\todoref}{\todo{ref}}

% Title Page
\title{Software Verification 2014 \\ Project Report}
\author{Roman Schmocker \\ 09-911-215}


\begin{document}

\maketitle


\section{Introduction}
This report covers the implementation and verification of the two sort algorithms BucketSort \cite{web:bucketsort} and QuickSort \cite{web:quicksort}.
The tools used to verify the code are AutoProof \cite{web:autoproof} and Boogie \cite{paper:boogie, web:boogie}.

The structure of the report follows the chronological order in which the tasks were completed.
The first section deals with the Boogie verification of the Quicksort algorithm.
Following that is a section about the AutoProof solution.
The last section compares the two tools and draws conclusions.

\section {Boogie}

In the Boogie part I only verified Quicksort due to time constraints.

The Boogie solution for Quicksort contains three procedures \lstinline!Swap!, \lstinline!Split! and \lstinline!QuickSort!.
All features manipulate a single global integer array where the sorting happens.

The main pieces needed for verification are the postconditions of all three features, as well as the loop invariants in \lstinline!Split!.
The specifications can be roughly split into four different groups to verify that
\begin{itemize}
 \item the result is sorted, 
 \item the result is a permutation of the input,
 \item the procedure does not modify the global array outside the specified range,
 \item and the minimum and maximum values within a specified range stay the same.
\end{itemize}

The last point was necessary because, although Boogie could verify that the result is a permutation of the original array,
it was not able to infer that the minimum and maximum values stay the same.
A workaround was therefore to keep track of these values explicitly by extending the interface of \lstinline!QuickSort! to get the boundary values
and add all these contracts which make sure the boundary is maintained.

Another challenge was to show that the elements of the final array are the same as in the input array.
An extensive literature study \footnote{Typing ``boogie sort algorithm'' in Google.} showed that the solution is to 
keep track of the permutation explicitly \cite{web:boogie:bubblesort}.

To do this I added two more global arrays, such that in the final solution there are three of them:
\begin{itemize}
 \item The initial input array. This array is never modified.
 \item The sorted output array.
 \item A permutation array to map the sorted array to the input array.
\end{itemize}

Using this system I was able to set up a global invariant that the permutation is always a valid mapping.
This invariant can be found in basically every precondition, postcondition and loop invariant, and it's the reason why the permutation proof became very simple in the end.

\section{Eiffel}

\subsection{Implementation}

The implementation of the features in \lstinline!SV_LIST! was pretty straightforward.
The problem however was, as I later realized, that the implementation wasn't always suited to verification.
Therefore the relevant sort features like Quicksort and Bucketsort had to be adapted several times 
and the final implementation carries some overhead to keep track of state needed for verification.

\subsection{Specification}

Due to my experience with Eiffel programs, specifying pre- and postconditions was easy.
The only challenge were the sorting algorithms, where the specification should not just include that the array is sorted,
but also that the final array contains the same elements as the input array.
Otherwise one could just return a zero-filled array as a valid result.
I solved this the same way as in the Boogie solution, by specifying that the result is a permutation of the input array.

The verification phase however brought some unplanned changes again.
I had to add some specifications, such as the number of elements staying the same during a sort, 
as AutoProof wasn't able to infer that from the postcondition about permutation only.

\subsection{Verification}

\subsubsection{General featuers}
The class \lstinline!SV_LIST! was shipped with many small features such as \lstinline!extend! or \lstinline!count!.
These were very easy to prove, often not even requiring any further specification besides pre- and postconditions.

\subsubsection{Quicksort}

The Quicksort algorithm is fully verified in AutoProof.
Besides proving that the output is sorted, I was also able to prove that the final result is a permutation of the input array.

The algorithm selects the first element in the input as a pivot and then distributes the remaining elements into two arrays (\lstinline!left! and \lstinline!right!).
It then recursively calls \lstinline!quick_sort! on both elements and merges the two results.

At the core of the verification for Quicksort are the loop invariants during the split phase.
There are two groups of invariants.
The first group is required to verify the \lstinline!sorted! postcondition.
It basically states that all elements smaller than the pivot element are in the \lstinline!left! array, whereas the other ones are in the \lstinline!right! array.

The second group of invariants keeps track of the \lstinline!permutation! postcondition.
To be able to verify permutation I had to introduce a new array \lstinline!control!, where all elements from the input are just inserted one by one.
The invariants then state that the concatenation of \lstinline!left!, \lstinline!right! and the pivot element is a permutation of the \lstinline!control! array,
and that the \lstinline!control! array will eventually be equal to the input array.

The main problem that I had during development was that AutoProof had some peculiar ideas about what could be a permutation.
At some point I was able to verify that the input array is \emph{equal} to a local array, and the local array is a permutation of the sorted array.
AutoProof however couldn'd draw the conclusion that therefore the sorted array is a permutation of the input array.
Thanks to a tip from Julian Tschannen I could finally resolve the issue by using a slightly different check for sequence equality.

I also encountered the same problem in AutoProof which I already had in Boogie:
Even though I could prove that two arrays contain the same elements (using permutations), the tool was not able to infer that the range of possible values stays the same.
To solve this I implemented the workaround from Boogie in Eiffel as well.
This means that the Quicksort implementation gets the boundary values as an argument and has to make sure that they're maintained throughout the call.

Overall, the verification of Quicksort in AutoProof was very tough.
In fact during my first attempt I stopped with the AutoProof solution and started working on Boogie.
After that I had a slightly better idea about what's going on in the background, and I was finally able to proof the algorithm correct.

\subsubsection{Bucketsort}

After a successful verification of Quicksort, implementing and verifying Bucketsort was very easy.
The algorithm is very similar to Quicksort.
It also consists of splitting up the array into disjoint parts and then calling a sort algorithm on them.
In this case the function didn't even have to be recursive, which simplified things a little.

I encountered one problem when trying to verify that the array is still sorted after calling \lstinline!quick_sort! on each of the buckets.
The solution was to introduce a new \lstinline!check! instruction which states that the ranges of values within each bucket are still disjoint.
This apparently helped the Boogie backend to trigger the right axiom, such that it was able to verify the algorithm.

\section {Conclusion}

During this project I was able to fully verify Quicksort in both AutoProof and Boogie, and Bucketsort in AutoProof only.
In my opinion Boogie is a little bit simpler to use because the language is fully designed with verification in mind
and it's easier to figure out what is going on during the verification process.

AutoProof on the other hand has a higher level of abstraction.
However I think that exactly this abstraction makes it harder to use AutoProof, 
because when something goes wrong you often have no idea and no hint about what causes the problem.
Another problem which mostly happened in AutoProof is that I often got a timeout error, which can be very frustrating.

An advantage of AutoProof on the other hand is that it nicely integrates into the Eiffel programming language.
This allows to at least partially prove some classes and features in a convenient and mostly automatic way
and use them in real software, instead of just verifying a sort algorithm for a student project.

In both tools there's very little help provided when debugging a failed assertion.
Getting it to verify therefore often involves hours of trial-and-error programming.
I think for a static verifier to succeed in practice it needs to at least provide a counterexample 
or say why it couldn't verify a statement, but this might be infeasible to do.

In my opinion static verification of programs is not yet very usable in practice,
because it can be very hard to prove even simple programs which programmers can easily reason about.
There's also the problem that the specification itself may be wrong with respect to the requirements, 
and the verifier thus happily verifies a buggy program.
Therefore I think verification can at most serve as a complement to traditional software quality assurance techniques.

\begin{flushleft}
{{{
\bibliographystyle {plain}
\bibliography {./references}
}}}
\end{flushleft}


\todos

\end{document}          
 
